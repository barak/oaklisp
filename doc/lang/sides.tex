% This file is part of Oaklisp.
%
% This program is free software; you can redistribute it and/or modify
% it under the terms of the GNU General Public License as published by
% the Free Software Foundation; either version 2 of the License, or
% (at your option) any later version.
%
% This program is distributed in the hope that it will be useful,
% but WITHOUT ANY WARRANTY; without even the implied warranty of
% MERCHANTABILITY or FITNESS FOR A PARTICULAR PURPOSE.  See the
% GNU General Public License for more details.
%
% The GNU GPL is available at http://www.gnu.org/licenses/gpl.html
% or from the Free Software Foundation, 59 Temple Place - Suite 330,
% Boston, MA 02111-1307, USA


\chapter{Side Effects} \label{sides}

The treatment of side effects in Oaklisp is modelled on that
of T.  The salient feature of this approach is the use of reversible
access procedures to perform side effects on composite data structures
and anonymous storage cells.

\section{Assignment}

Side effects on variables and objects are performed with the
\df{set\protect\bang} special form, which combines the functionality of the
\df{setq} and \df{setf} forms found in other Lisps.

\sform{set\protect\bang}{location new-value}
\doc{Changes the value of \emph{location} to \emph{new-value}, which is then
returned as the value of the expression.

If \emph{location} is a symbol, then it is interpreted as a variable
name.  The variable must have been previously defined in some lexical binding
contour or locale.

If \emph{location} is a list, then it is interpreted as a reference to
a settable access operation.  For example, \texttt{(set! (car foo) 'bar)}
means the same thing as \texttt{(rplaca foo 'bar)} in Common Lisp.}

\lo{setter}{operation}
\doc{Takes a settable access operation and returns the corresponding
alteration operation.}

\section{Locatives}

\df{locative} is an Oaklisp type that is similar to the pointer types
found in procedural languages such as C or Pascal.  Locatives are
created and dereferenced by the following constructs.

\sform{make-locative}{location}
\doc{Returns a locative that points to \emph{location}, which must be
a variable or a list with the form of a call on a locatable access
operation.}

\lo{locater}{operation}
\doc{Takes a locatable access operation and returns the corresponding
locative-making operation.}

\lo{contents}{locative}
\doc{Returns the contents of the location which is referenced by 
\emph{locative}.  Since \df{contents} is a settable operation, side effects
can be performed on locations through locatives.  For example,
\texttt{(set! (contents (make-locative (car \emph{foo})))
'\emph{bar})} has the
same effect as \texttt{(set! (car \emph{foo}) '\emph{bar})}.}


\section{Operation Types}

Since operations are objects, they are classified into types
according to the operations which can be performed on them.  The types
discussed here can generate side-effecting operations from access
operations.

\ty{operation}
\doc{This is the generic operation type that is a component of all other
operation types.}

\ty{settable-operation}
\doc{An access operation is settable if side effects can be performed
through it.  Settable operations respond to \df{setter}.}

\ty{locatable-operation}
\doc{An access operation is locatable if it retrieves information
from a single physical location.  Locatable operations respond to
\df{setter} and \df{locater}.}

\section{Modification Forms}

See chapter 6 of {\it The T Manual} for a description of the
following forms.

\sform{swap}{location new-value}
\sform{modify}{location procedure}
\sform{modify-location}{location continuation}

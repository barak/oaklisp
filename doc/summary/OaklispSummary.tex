\documentclass[12pt]{article}
\usepackage[letterpaper,margin=.75in]{geometry}
\usepackage{alltt}
\addtolength{\parskip}{\baselineskip}  % causes a blank line between paragraphs
\setlength{\parindent}{0in}

\usepackage{fancyhdr}
\pagestyle{fancy}
\renewcommand{\headrulewidth}{0in}
\rfoot{\small Page \thepage\ of \pageref{lastpage}}
\cfoot{}
\lfoot{\small rev. 12/26/14}

\begin{document}

\centerline{\Huge Oaklisp Summary}
\centerline{Blake McBride (blake@mcbride.name)}

\medskip

{\large \bfseries Startup:}
\smallskip
\begin{verbatim}
    oaklisp  [--world <file>]  [--dump <file>]  [-G]

where:
        --world indicates what world to load. Default is oakworld.bin
                that comes with the system

        --dump provides a file name where a new world will be dumped
               when the system exits

        -G  option to perform a GC prior to dumping a world
\end{verbatim}

{\large \bfseries Exiting the system:}

\begin{verbatim}
    (exit)

    Use ^D to exit error states
\end{verbatim}

{\large \bfseries File types:}

\smallskip

\hspace{.75in}
\hbox{\begin{tabular}{|l|l|}\hline
File ending & Description \\\hline
oak & Oaklisp source file \\
&\\
oa & compiled Oaklisp file (fasl) \\
&\\
bin & Oaklisp world --- binary representation of\\
    & an entire Oaklisp system representing all \\
    & objects in the running environment.\\\hline
\end{tabular}}

\smallskip

{\large \bfseries Loadind and compiling files:}
\smallskip
\begin{verbatim}
    (load "myfile")  ; will load "myfile.oa" or "myfile.oak"

    (compile-file #*current-locale "myfile") ; compiles myfile.oak to myfile.oa
\end{verbatim}

\newpage

{\Large \bfseries Data Types}

Use {\ttfamily (get-type x)} to get the type of an item.

\hspace{.75in}
\hbox{\begin{tabular}{|l|l|l|}\hline
Type & Comment & Example \\\hline
null-type & null list \& false & nil or \#f or {\ttfamily \symbol{13}}() \\
truths & true & t or \#t \\
&&\\
fixnum &  decimal & 48  \\
\ \ \&  & binary & \#b1011 \\
bignum & binary & \#2r1011 \\
 & octal & \#o755 \\
 & hex & \#xD5 \\
&&\\
fraction & decimal & 2/3 \\
 & hex & \#xDE/FF \\
 & floats are converted to fraction & 3.141 \\
&&\\
character & letter `M' & \#\verb+\+M \\
 & & \#\verb+\+Space \\
 & & \#\verb+\+Newline \\
 & & \#\verb+\+Backspace \\
 & & \#\verb+\+Tab \\
 & & \#\verb+\+Return \\
 & & \#\verb+\+Page \\
&&\\
symbol & case insensitive (up-cased) &  {\ttfamily \symbol{13}}symb \\
 & case sensitive &   {\ttfamily \symbol{13}}$\mid$symb$\mid$ \\
 & embedded `(' &   {\ttfamily \symbol{13}}sy\verb+\+(mb \\
&&\\
string & with embedded \verb+"+ & ``hello\verb+\"+ there'' \\
&&\\
list && {\ttfamily \symbol{13}}(hello there) \\
&&\\
simple-vector && \#(3 4 hello) \\
&&\\
type & root of all types & type \\
     & root of all objects & object \\
\hline
\end{tabular}}

\newpage


{\Large \bfseries List Operations}

\bigskip

\hbox{\begin{tabular}{|l|l|l|l|l|l|}\hline
Type & Comment & Sp & D & Example & Result \\\hline
car & first element & fast & no & (car {\ttfamily \symbol{13}}(A B C)) & A \\
cdr & rest of list & fast & no & (cdr {\ttfamily \symbol{13}}(A B C)) & (B C) \\
c????r & shorthand for & fast & no & & \\
 & multiple car \& cdr's & & & & \\
 & & & & & \\
cons & add a list node & fast & no & (cons {\ttfamily \symbol{13}}A {\ttfamily \symbol{13}}(B C)) & (A B C) \\
list & create a list & fast & no & (list {\ttfamily \symbol{13}}A {\ttfamily \symbol{13}}B {\ttfamily \symbol{13}}C) & (A B C) \\
append & append lists & slow & no & (append {\ttfamily \symbol{13}}(A B) {\ttfamily \symbol{13}}(D) {\ttfamily \symbol{13}}(G H)) & (A B D G H) \\
length & & slow & no & (length {\ttfamily \symbol{13}}(A B C)) & 3 \\
nth & & slow & no & (nth {\ttfamily \symbol{13}}(A B C) 1) & B \\
first & same as (nth 0 x) &  &  & & \\
second & same as (nth 1 x) &  &  & & \\
third & same as (nth 2 x) &  &  & & \\
etc. & &  &  & & \\
 & & & & & \\
last & & slow & no & (last {\ttfamily \symbol{13}}(A B C)) & (C) \\
 & & & & & \\
 & Given: & & & & \\
 & (setq lst {\ttfamily \symbol{13}}(A B C)) & & & & \\
 & replace car cell & fast & yes & (set! (car lst) {\ttfamily \symbol{13}}X) & (X B C) \\
 & replace cdr cell & fast & yes &  (set! (cdr lst) {\ttfamily \symbol{13}}X) & (A . X) \\
\hline
\end{tabular}}

\bigskip
\begin{description}
  \item{Sp} = Speed
  \item{D} = Destructive
\end{description}

\newpage


{\Large \bfseries Comments}

\smallskip

Semi-colon `;' introduces a comment.  Comments extend from the semi-colon to the end of the line.

\bigskip

{\Large \bfseries Evaluation and Quotes}

\smallskip

Lists are recursively evaluated.  The first element of a list is the function and the remainder are arguments to the function.  For example:

\begin{verbatim}
    (func arg1 arg2 ...)
\end{verbatim}

Quoting stops evaluation, so


\begin{verbatim}
    (quote (abc def ghi))
\end{verbatim}

would not attempt to evaluate \emph{abc}, \emph{def}, or \emph{ghi}.  The ``{\ttfamily \symbol{13}}'' symbol acts as a shorthand for \emph{quote}, so the following is equivelent to the previous statement.

\hspace*{.3in}{\ttfamily \symbol{13}}\verb+(abc def ghi)+






\bigskip

{\Large \bfseries Predicates}

\bigskip

\hbox{\begin{tabular}{|l|l|}\hline
function & Returns \emph{NIL}, or \emph{\#T} for \\\hline
null? & null \\
symbol? & symbols \\
atom? & non-list items \\
list? &  list items (including \emph{NIL}) \\
number? & all numbers \\
integer? & integers \\
string? & strings \\
vector? & vectors \\
eq? & same object \\
equal? & structurally similar objects \\
       & or same number\\
\hline
\end{tabular}}

\newpage

{\Large \bfseries Logical Operators}

\bigskip

\hbox{\begin{tabular}{|l|l|l|l|}\hline
function & Returns & Example & Comment \\\hline
not & nil or \#T & (not exp) & Logical inverse \\
and & last exp or nil & (and exp1 exp2 ...) & Stops evalution on first null expression \\
or  & first non-nil exp & (or exp1 exp2 ...)  & Stops evalution on first non-null expression \\
\hline
\end{tabular}}

\emph{and} and \emph{or} can be used as conditionals too.  For example:

\begin{verbatim}
    (and exp1 exp2 exp3)
\end{verbatim}

is the same as

\begin{verbatim}
    (if exp1
        (if exp2 exp3))
\end{verbatim}



\bigskip

{\Large \bfseries Block}

\bigskip

\parbox[t]{3.4in}{
(block\\
\hspace*{2em}    exp1\\
\hspace*{2em}    exp2\\
\hspace*{2em}    ...)
}\parbox[t]{3.5in}{
 Execute each expression and return the value of the last expression.
}

\bigskip

{\Large \bfseries Local Variables}

\bigskip

\parbox[t]{3.4in}{
(let ((var1 val1) ; initialize \emph{var1} to \emph{val1}\\
\hspace*{2.5em}(var2 nil) ; initialize \emph{var2} to \emph{nil}\\
\hspace*{2.5em}...)\\
\hspace*{4em}    exp1\\
\hspace*{4em}    exp2\\
\hspace*{4em}    ...)
}\parbox[t]{3.5in}{
 \emph{LET:}\ \ Creates local variables and executes the expressions in their context.  Returns the value of the last expression.
}

\parbox[t]{3.4in}{
(let* (vars)  exp1 exp2 ...)
}\parbox[t]{3.5in}{
 \emph{LET*:}\ \ Sames as \emph{LET} except the variables are assigned sequentially.  Previously defined variables may be used in subsequent variable initializations.
}


\newpage

{\Large \bfseries Defining Functions}

\bigskip


\parbox[t]{3.4in}{
(define (function-name arg1 arg2 arg3 ...)\\
\hspace*{4em}    exp1\\
\hspace*{4em}    exp2\\
\hspace*{4em}    ...)\\
\\
(define function-name\\
\hspace*{4em}(lambda (arg1 arg2 arg3 ...)\\
\hspace*{4em}    exp1\\
\hspace*{4em}    exp2\\
\hspace*{4em}    ...))
}\parbox[t]{3.5in}{
 \emph{DEFINE:}\ \ Define function named \emph{function-name} with specified arguments, and run the expressions in that context.
}

\bigskip

{\Large \bfseries Conditionals}

\smallskip

In Oaklisp, conditions are considered true if they are any value other
than \emph{NIL} or \emph{\#F}.  Therefore, \emph{NIL} and \emph{\#F} are considered as \emph{false},
and all other values are treated as \emph{true} conditions.


\parbox[t]{3.4in}{
(if test\\
\hspace*{2em}    exp1\\
\hspace*{2em}    [exp2])
}\parbox[t]{3.5in}{
 \emph{IF:}\ \ If \emph{test} expression is \emph{true}, return \emph{exp1}.
 \emph{exp2} is optional.  If \emph{test} is \emph{false}, return \emph{exp2} or \emph{undefined-value}.
}

\parbox[t]{3.4in}{
(cond\\
\hspace*{2em}    (test1 exp1 exp2 ...)\\
\hspace*{2em}    (test2 exp1 exp2 ..)\\
\hspace*{2em}    ...\\
\hspace*{2em}    (else exp1 exp2 ...))
}\parbox[t]{3.5in}{
 \emph{COND:}\ \ Evalute each \emph{testN} until one is \emph{true}.
 If a \emph{true} test is found, its expressions are evaluated and the
 result of the last one is returned.  The \emph{else} test is always \emph{true}.
}



\bigskip

{\Large \bfseries Math Predicates}

\bigskip

\hbox{\begin{tabular}{|l|l|}\hline
Function & Description \\\hline
(number? n) & Is \emph{n} a number \\
(zero? n) & \\
(odd? n) & \\
(even? n) & \\
(negative? n) & \\
(positive? n) & \\
(= a b) & \\
(!= a b) & \\
(\verb+<+ a b) & \\
(\verb+>+ a b) & \\
(\verb+<+= a b) & \\
(\verb+>+= a b) & \\
\hline
\end{tabular}}

\newpage

{\Large \bfseries Numeric Operators}

\bigskip

\hbox{\begin{tabular}{|l|l|}\hline
Function & Description \\\hline
+ - * / & Math functions \\
(abs n) & Absolute value \\
(1+ n) & \\
(quotient a b) & integers only \\
(minus n) & \\
(floor n) & \\
(ceiling n) & \\
(round n) & \\
(truncate n) & \\
(modulo a b) & \\
(remainder a b) & \\
(expt n p) & \\
(max a b c ...) & \\
(min a b c ...) & \\
(numerator f) & \\
(denominator f) & \\
\hline
\end{tabular}}


\bigskip

{\Large \bfseries String Predicates}

\bigskip

\hbox{\begin{tabular}{|l|l|}\hline
Function & Description \\\hline
(string? s) & is \emph{s} a string \\
(= a b) & \\
(equal? a b) & \\
(!= a b) & \\
(\verb+<+ a b) & \\
(\verb+>+ a b) & \\
(\verb+<+= a b) & \\
(\verb+>+= a b) & \\
\hline
\end{tabular}}

\bigskip

{\Large \bfseries String Operators}

\bigskip

\hbox{\begin{tabular}{|l|l|}\hline
Function & Description \\\hline
(length s) &  \\
(nth s i) & 0 origin\\
(upcase s) &\\
(downcase s)&\\
(reverse s)&\\
(append s1 s2)&return an appended copy\\
(copy s)&\\
(subseq s beg len)&\\
\hline
\end{tabular}}

\newpage


{\Large \bfseries Macros}

\smallskip

\begin{alltt}
    (define-syntax (add a b)
        \symbol{18}(+ ,a ,b))

    (gensym \symbol{13}tmp)
\end{alltt}

{\Large \bfseries Named let}

\smallskip

\begin{alltt}
    (let loop ((v 0))
        (if (!= v 10)
            (block
                (print v standard-output)
                (newline)
                (loop (1+ v)))))
\end{alltt}


{\Large \bfseries Printing}

\smallskip

\begin{alltt}
    (print x standard-output)
    (newline)
    (format <stream> <fmt> <arg1> <arg2> ...)
        <stream> = #T = standard-output
                   #F = return a string
\end{alltt}

\newpage

{\Large \bfseries Classes and Types}

In classical object-oriented terminology, the following terms apply:

\bigskip

\hbox{\begin{tabular}{|l|l|}\hline
Name & Description \\\hline
Class & describes the structure and functionality associaed with instances of it \\
& (e.g.\ instance variables and instance methods) \\
Meta Class & describes the structure and functionality of the class object \\
& (e.g.\ class variables and class methods) \\
Instance & a unique instance of a class \\
Object & a class, metaclass, or instance \\
\hline
\end{tabular}}


In Oaklisp, classes and types are the same thing.  So, every type in
Oaklisp, such as list, symbol, string, number, fraction, etc., are all
classes.  Also, all types and instances are objects.  In Oaklisp, all
types and objects are first-class, this means they can be passed to
functions, assigned to variables, etc..  They are all treated the
same.


Oaklisp has two root classes/types named
\emph{Object} and \emph{Type}.  \emph{Type} is the root of all types, and \emph{Object} is the
root of all objects. 

For purposes of this and following examples, the following naming scheme shall be used:


\bigskip

\hbox{\begin{tabular}{|l|l|}\hline
Name & Description \\\hline
Type & the Type type \\
type & a type \\
Object & the Object type \\
object & an object \\
iv* & instance variable \\
cv* & class variable \\
i* & an instance object \\
im* & instance method \\
cm* & class method \\
\hline
\end{tabular}}


\bigskip

{\large \bfseries Classes -- Simple}

By ``simple'' we mean without meta classes.

\textbf{Creating a type or class:}

Format:

\begin{alltt}
   (define <type-name>
           (make Type <instance variable names> <superclass list>))
\end{alltt}

\newpage

Example:

\begin{alltt}
   (define MyClass (make Type \symbol{13}(iv1 iv2) (list Object)))
\end{alltt}

\textbf{Creating an instance of a type:}

Format:

\begin{alltt}
   (make <your-type>)
\end{alltt}

Example:

\begin{alltt}
   (define i1 (make MyClass))
\end{alltt}

\textbf{Inspecting an object:}

Format:

\begin{alltt}
   (describe <object>)
\end{alltt}

Example:

\begin{alltt}
   (describe MyClass)
   (describe i1)
\end{alltt}


\textbf{Creating instance methods:}

Format:

\begin{alltt}
   (define-instance <method-name> operation)
   (add-method (<method-name>
                     (<type name>  <instance variables this method will access>)
                     self  <method argument list>)
               <method code>
               ...)
\end{alltt}

\newpage

Example:

\begin{alltt}
   (define-instance imSet-iv1 operation)
   (add-method (imSet-iv1 (MyClass iv1) self a)
               (set! iv1 a))
  
   (define-instance imGet-iv1 operation)
   (add-method (imGet-iv1 (MyClass iv1) self)
               iv1)

   (imSet-iv1 i1 33)
   (imGet-iv1 i1)
\end{alltt}


\textbf{Instance constructor / initialization:}

Format:

\begin{alltt}
   (add-method (initialize  (<type name>  <instance variables being accessed>)
                            self  <argument list>)
                     <initilization code>
                     ...
                     self)
\end{alltt}

Example:

\begin{alltt}
   (add-method (initialize (MyClass iv1 iv2) self arg1)
        (set! iv1 arg1)
        (set! iv2 44)
        self)

    (define i2 (make MyClass 88))
    ; in this case
    ; i2->iv1 would be 88
    ; i2->iv2 would be 44
\end{alltt}

\newpage

{\large \bfseries Classes -- Complex}

By ``complex'' we mean with the addition of meta classes.  This gives us class variables and class methods.

\textbf{Creating a type or class:}

Creating a type with a meta-type involves the creation of two types; one the meta type and the other the type.
The type created must be an instance of the meta type.

Adding initializers and instance methods are as before.  The only difference is we now have the ability
to define and use class methods and variables.  Those will be shown.

Format:

\begin{alltt}
   (define <meta-type-name>
           (make Type <class variable names> <meta superclass list>))
   (define <type-name>
           (make <meta-type-name>  <instance variable names>  <superclass list>))
\end{alltt}

Example:

\begin{alltt}
    (define metaMyClass2 (make Type \symbol{13}(cv1 cv2) (list Type)))
    (define MyClass2 (make metaMyClass2 \symbol{13}(iv1 iv2) (list Object)))
\end{alltt}

\textbf{Class methods:}

Format:

\begin{alltt}
   (define-instance <method-name> operation)
   (add-method (<method-name>
                     (<meta type name>  <class variables this method will access>)
                     self  <method argument list>)
               <method code>
               ...)
\end{alltt}

\newpage

Example:

\begin{alltt}
   (define-instance cmSet-cv1 operation)
   (add-method (cmSet-cv1 (metaMyClass2 cv1) self a)
               (set! cv1 a))
  
   (define-instance cmGet-cv1 operation)
   (add-method (cmGet-cv1 (metaMyClass2 cv1) self)
               cv1)

   (cmSet-cv1 MyClass2 33)
   (cmGet-cv1 MyClass2)
   (describe MyClass2)  ; to see the contents of the class
\end{alltt}


\label{lastpage}
\end{document}

